\documentclass{article}
\usepackage[icelandic]{babel}
\usepackage[T1]{fontenc}
\usepackage[utf8]{inputenc}
\usepackage{graphicx}
\usepackage{wrapfig}
\usepackage{subfig}
\usepackage{float}
\usepackage{todonotes}
\usepackage{color}

%Leitað er af myndum í eftirfarandi möppum
\graphicspath{{./technical/}{./sprettir/}}
\linespread{1.5}
\setcounter{tocdepth}{2}

\usepackage{graphicx}
\usepackage{wrapfig}
\usepackage{float}

\begin{document}

\tableofcontents
\newpage

\section{Inngangur}
Verkefnið snérist um að búa til kerfi sem greinir
breytingar á gagnasettum í kerfum DataMarket og flagga
frávik frá eðlilegri þróun tiltekinnar tímalínu, sem
gætu bent til þess að um áhugaverðan atburð sé að ræða.

Fyrirhugað er að kerfið okkar verði keyrt innan kerfis
Datamarket og er hugsa þannig að hægt sé að keyra það á
gögnin þeirra hvenær sem er.

Upprunalega var kerfið okkar hugsa þannig að það myndi
athuga hvort ný gögn sem væri verið að bæta við
gangasett Datamarket væru áhugaverð eður ei. Hinsvegar
varð okkur ljóst í þróunnarferlinu að kerfið okkar gæti
gert meira en unnið eingöngu með nýjustu uppfærslurnar
og varð því kerfið okkar yfirgripsmeira ein upprunalega
var hugsað.

\newpage
\section{Lýsing verkefnis}
Sú þjónusta sem Datamarket veitir er að taka sama töluleg gögn frá opinberum
stofnunum og einkaaðilum og gerir þau aðgengileg á einum stað sem og birta þau á
auðskiljanlegan máta. Gagnsafn Datamarket er gríðarstórt og fer ört vaxandi. Frá
og með 25. janúar 2011 samanstóð það af meira en 13.000 gagnasettum sem innihélt
næstum 100 milljón tímalínur. Það gefur augaleið að það er ógerlegt að fara
handvirkt í gegnum þetta magn af gögnum til að finna áhugaverða atburði því ætti
kerfið okkar að vera kærkomin viðbót við kerfi Datamarket.

Hvert gagnasett hjá DataMarket inniheldur tímaraðir sem
sýna þróun tölulegra stærða yfir tíma. Á hverjum degi
eru tugir, eða jafnvel hundruð slíkra gagnasetta
uppfærð í kerfinu. Hvert gagnasett inniheldur svo að
lágmarki eina, en allt að nokkurþúsund tímaraðir. Sú
þróun sem ein tímaröð sýnir getur verið allt frá vel
þekktum stærðum, s.s. verðbólgu, atvinnuleysi eða
hitastigi í Reykjavík, til mjög sérhæfðra eða jafnvel
undarlegra hluta, eins og innflutningsverðmæti
leðurvara frá Brasilíu!

Mikið er fylgst með þessum algengustu stærðum og
stórvægilegar breytingar í þeim rata iðulega í fréttir
umsvifalaust. En mjög markverð, áhugaverð eða jafnvel
varasöm þróun getur líka birst í stærðum sem fáir
fylgjast með og enginn tekur etv. eftir.

Markmiðið með þessu verkefni er að greina þessi tilvik
sjálfkrafa þannig að hægt sé að beina athygli notenda
DataMarket að þeim. Þannig mætti t.d. hugsa sér að
dregnir yrðu fram á forsíðu DataMarket tenglar á þróun
sem kerfið telur um markverða atburði að ræða.

Verkefnið snýst s.s. um að útbúa aðferðafræði sem er
líkleg til að grípa tilvik af þesu tagi. Tiltölulega
auðvelt er að útbúa mjög gróf algrím sem myndu grípa
mörg tilvik, en líklega einnig leiða af sér töluvert
magn “false positives”. Hugsunin er að byrja á að útbúa
slíka einfalda frumgerð og endurbæta svo aðferðafræðina
eftir því sem reynsla, hugmyndaauðgi og tími gefa
tilefni til.

\newpage

\section{Framvinda}
\subsection{Sprettir}
Verkefnið okkar var þannig lagt upp af DataMarket, að þeir vildu í fyrsta lagi fá heilstætt kerfi sem skilaði niðurstöðum. 
Þær niðurstöður þyrftu þó ekki að verja mjög marktækar og byggja á flóknum reikniaðferðum. Því næst áttum við að þróa reikniaðferðirnar eins 
mikið og mögulegt væri, á þeim tíma sem við höfðum. Þetta verkefni var því mjög opið og ljóst að mikilvægt væri 
af okkar hálfu að halda góðri yfirsýn allan tímann, svo að ekki yrði farið lengra í bætingum en svo að það myndi nást að fullklára þær 
aðferðir sem við ætluðum að hafa í kerfinu, í tæka tíð. Því lögðum við upp með að vinna verkið í tveimur fösum. Áætlunin var þannig að skipuleggja
einungis fyrri fasan í upphafi. Byrja að kynna okkur meðfram þeim fasa hvað við gætum mögulega gert í seinni fasanum, en taka ekki ákvarðanir 
um hvað yrði gert fyrr en í upphafi seinnifasa.

\subsubsection{Sprettur 0}
Sprettur 0 var settur upp sem undirbúnings sprettur. Ekki voru sögur eins og í hefbundnum spretti, heldur notuðum við þennan tíma til að 
skipuleggja verkáætlun, gerðum áhættugreiningu og unnum að ýmsum öðrum undirbúiningi.
\subsubsection{Sprettur 1}
Í fyrsta spretti fórum við í að kynna okkur tæknileg atriði sem við ætluðum að nota.
Flest þekktum við, en höfðum ekki unnið með það í Python áður. Þetta voru hlutir eins og prófandrifin þróun, REST-ful þróun og hvernig 
unnið er með JSON skrár. Einnig settum við upp grunn að kerfinu sem gat sótt gögn frá DataMarket. Náðum ekki alveg að kára það sem við lögðum 
upp með og urðum að færa hluta úr sögum yfir í næsta sprett.
\begin{figure}[H]
  \centering
  \includegraphics[width=0.72\textwidth]{Sprettur1_Burndown.png}
  \caption{Sprettur 1 Burndown}
\end{figure}

\subsubsection{Sprettur 2}
Í spretti 2 fundum við áhugaverð gögn hjá Datamarket settum upp staðbunda gátt(e.service stub),svo að það myndi ekki hindra okkur í þróun ef 
aðgangur að gagnagrunni Datamarket væri ekki til staðar. Þetta var hluti af okkar áhættugreiningu (VÍSUN í KAFLA). 
Þá bjugum við til fyrstu reikniaðferðirnar, meðaltal miðgildi og staðalfrávik, og létum kerfið sækja gögn og skila niðurstöðum.
Þá áttum við fundi með sérfræðingum á sviði stærðfræði og tölfræði, til að ýta úr vör hugmyndavinnu fyrir seinni fasa.
\begin{figure}[H]
 \centering
 \includegraphics[width=0.72\textwidth]{Sprettur2_Burndown.png}
 \caption{Sprettur 2 Burndown}
\end{figure}
\subsubsection{Sprettur 3}
Þegar hér var komið við sögu vorum við farnir að finna ákveðinn takt í áætlanagerð og sáum betur hvað við gátum gert ráð fyrir að klára mikið 
í einum sprett. Setum upp aðgerðarsöfn(e. libraries) sem voru til þess fallin að aðstoða okkur við útreikninga. Það tók þó umtalsvert meiri tíma en við gerðum
ráð fyrir. Þá bættum við staðbundnu gáttina og kynntum okkur aðgerðarsöfnin.
\begin{figure}[H]
 \centering
 \includegraphics[width=0.72\textwidth]{Sprettur3_Burndown.png}
 \caption{Sprettur 3 Burndown}
\end{figure}
\subsubsection{Sprettur 4}
Spretturinn fór í að setja upp framenda á kerfið og forma skýrsluna sem það skilar af sér. Þegar við vorum svo farnir að nota aðgerðarsöfnin 
kom á daginn að einföldu aðferðirnar okkar voru mun betri en vonir stóðu til. Einnig komumst við að því að það hafa aðrir beitt þessum aðferðum 
og kallast þær Bollinger bönd ( SJÁ KAFLA UM BOLLINGER BÖND ). 

Við gátum nú sýnt þeim hjá Datamarket niðurstöður úr kerfinu okkar og við það
urðu til fleiri hugmyndir um hvernig hægt væri að nýta það ( SJÁ LÝSING
VERKEFNIS ).

Við skipulögðun sprettinn létt og kláruðum snemma þar sem próf voru framundan.
\begin{figure}[H]
 \centering
 \includegraphics[width=0.72\textwidth]{Sprettur4_Burndown.png}
 \caption{Sprettur 4 Burndown}
\end{figure}

\subsubsection{Sprettur 5}
Síðasti sprettur í fasa 1. Funduðum með DataMarket og ákváðum endanlegt form á flöggum sem er skilað. Gengum frá lausum endum, og kóði yfirfarinn 
(e. refactor) og útgáfa 1.0 varð til. Hugmyndavinna fyrir fasa 2 komin á fullt.

\begin{figure}[H]
 \centering
 \includegraphics[width=0.72\textwidth]{Sprettur5_Burndown.png}
 \caption{Sprettur 5 Burndown}
\end{figure}

\subsubsection{Sprettur 6}
Við höfðum sankað að okkur mikið af upplýsingum og höfðum ákveðnar hugmyndir um hvað okkur langaði að gera. Það krafðist þess hinsvegar að við 
þurftum að framkvæma prófanir til að sjá hvað myndi henta okkar kerfi best. Því skipulögðum við sprettinn létt og bættum í hann þegar líða tók á. 
Fljótlega komumst við þó að niðurstöðum um hvaða útfærslur við vildum nota ( SJÁ TÆKNIKAFLA ). Vinna við lokaskýrslu hófst. Í lok sprettsins 
vorum við komnir með verulega bættar reikniaðgeðir frá því sem var.

\begin{figure}[H]
 \centering
 \includegraphics[width=0.72\textwidth]{Sprettur6_Burndown.png}
 \caption{Sprettur 6 Burndown}
\end{figure}

\subsubsection{Sprettur 7}
Byrjað á lokafrágangi. Allar keyrslustillingar lesnar úr skrá og villumeðhöndlun yfirfarin. Sett upp log kerfi. Álagsprófanir framkvæmdar og 
niðurstöður úr þeim yfirfarnar. Fínstillingar á reikniritum í kjölfar prófanna og villur lagfærðar. 

\begin{figure}[H]
 \centering
 \includegraphics[width=0.72\textwidth]{Sprettur7_Burndown.png}
 \caption{Sprettur 7 Burndown}
\end{figure}

\subsubsection{Sprettur 8}
Skýrslugerð í aðalatriði í lokasprett. Lokafrágangur á kóða. Læddum inn einni bætingu á reikniaðferðum og prófuðum uppá nýtt.
\begin{figure}[H]
 \centering
 \includegraphics[width=0.72\textwidth]{Sprettur3_Burndown.png}
 \caption{Sprettur 8 Burndown RÖNG MYND}
\end{figure}


\newpage


\section{TækniSkýrsla}


\subsection{Fourier transformations} 

Fourier umbreytingar eru stærðfræðilegar aðgerðir
sem brjóta merki niður í tíðnir þess, slíkt gátum við nýtt okkur með því að láta
umbreyta tímalínu í tíðnir. 
Ef ákveðin tíðni er mjög áberandi í tíðniritinu, þá er að öllum líkindum
endurteknging með þeirri tíðni í tímalínunni. 
Sem dæmi um slíkt er tímalínan á mynd~\ref{fig:rafmagnsnotkun} sem sýnir
rafmagnsnotkun í ótilgreindu landi.


\begin{figure}[H]
  \centering
   
\subfloat[Tímalína]{\label{fig:rafmagnsnotkun}\includegraphics[
width=0.42\textwidth]{Rafmagnsnotkun}}                    
   
\subfloat[Tíðnirit]{\label{fig:fouriergraph}\includegraphics[
width=0.42\textwidth]{Fourier}} 
  \caption{Fourier Transformations}
  \label{fig:fourier}
\end{figure}

Það má sjá að tímalínan hefur ákveðnar endurtekningar,
nánar til tekið þá er mikil notkun á veturna en lítil á
sumrin.
Þessa föstu sveiflu í tímalínunni greinir Fourier og
skilar grafi á mynd~\ref{fig:fouriergraph}. 
Úr því er síðan hægt að greina að endurtekning á sér stað með 12 staka
millibili.




\subsection{Hlaupandi meðaltal (HM)}
\label{sec:running_average}
Í tölfræði er hlaupandi meðaltal, einnig kallað
fljótandi meðaltal, notað til greiningar á gagnamengi. 
Er það gert með því að mynda röð meðaltala úr
mismunandi hlutmengjum heildarmengisins.
Ef gefin er röð $Z$ talna og ákveðin stærð ramma
(hlutmengis) er $N$, þá er fljótandi meðaltal fundið
með því að fyrst reikna meðaltal 
talna úr sæti $0,1,\dots,N-1$. Þá er ramminn færður
fram um eitt sæti og meðaltal fundið af tölum í sætum
$1,2,\dots,N$. 
Þessi aðgerðarröð er endurtekin yfir alla talnaröðina,
$Z-N$ sinnum.  

\begin{center}
  $HM = \frac{x+(x+1)+\dots+(x+(N-1))}{N}$ 
\end{center}


Línan sem tengir svo saman öll meðaltölin er hið
fljótandi meðaltal þar sem hver punktur á línunni
samsvarar 
meðaltali viðkomandi hlutmengis í heildarmengi ganganna
sem verið er að meta. 
 
\subsection{Vegið hlaupandi meðaltal (VHM)}
\label{sec:weighted_running_average}
Hlaupandi meðaltal getur einnig notað ójöfn gildi fyrir
hvert stak á línunni.
Þá er um vegið meðaltal að ræða og er meðaltalinu gefin
einhver margfeldisáhrif eftir því hvar á línunni það er
staðsett. 
Það vægi breytist línulega. Summu margfeldi allra
gildana í tilteknum ramma er svo deilt með summu allra
mergfeldanna. 
Ef um ramma af stærð 10 er að ræða og gildin eru
margfölduð eftir sætisröð, þá fæst meðaltalið með eftirfarandi aðferð. 
\begin{center}
  $VHM = \frac{x+2(x+1)+3(x+2)+\dots+(10(N-1))}{1+2+\dots+10}$ 
\end{center}


\subsection{Bollinger bönd}
\label{sec:bollinger_bands}


Bollinger bönd eru upprunalega þróuð sem greiningartól
á þróun verðbréfaverða. 
Tilgangur þeirra er að veita viðeigandi skilgreiningu háum og lágum gildum. Einnig hefur þessari 
aðferð verið beitt á ýmis önnur vandamál með misgóðum niðurstöðum. \\

Bollinger bönd samanstanda af:
\begin{itemize}
  \item Miðband, $N-lotu$ einfalt hlaupandi meðaltal.
  \item Efra band, $N-lotu$ staðalfrávik margfaldað með $K$, fyrir ofan miðbandið ($HM + K\sigma$).
  \item Neðra band, $N-lotu$ staðalfrávik margfaldað með $K$, fyrir neðan miðbandið ($HM -K\sigma$).
\end{itemize}

%\begin{wrapfigure}{r}{7cm}
\begin{figure}][H]
 \begin{center}
  \includegraphics[width=.58\textwidth]{Bollinger.png} 
  \caption{Bollinger bönd merkt með Rauðum strikalínum} 
  \end{center}
\end{figure}
%\end{wrapfigure}


Bollinger bönd nýtast því vel til greininga á því hvar í
tímalínu óeðlileg hækkun eða lækkun á sér stað.






\section{útfærsla á Rýni}
\label{sec:implementation}


\subsection{vantar titil}
\label{sec:imp_our}

Útfærslan okkar byggist á því að ítra mismunandi stórar rammastærðir fyrir hverja tímalínu.
Allir púnktar í tímalínunni fá einkunn frá hverri rammastærð fyrir sig. 
Til þses að gefa púnkti einkunn notum við okkar eigin útgáfu af því sem kallast $\%b$. 
Útgáfan okkar gefur púnkti tölugildi á bilinu 0-1 ef hann er á milli Bollinger bandanna.
Ef púnkturinn liggur utan þeirra fær hann tölugildi sem samsvarar hlutfallslegri fjarlægð hans frá meðaltali rammans, 
sú tala er alltaf hærri en einn.

\begin{center}
  $\%b=\frac{abs((item - avg[index])}{(upperlim[index] - avg[index]))}$
\end{center}

{
  \color{red}
Hver rammastærð margfaldar sína útkomu úr formúlunni
við gildið sem punkturinn hefur fyrir.
}
Þannig gefum við punktum sem líta út fyrir að vera
snörp hækkun í litlum ramma lækkað vægi þegar hann
lendir innan bandanna í stærri ramma
og telst því til eðlilegrar hegðunar þegar horft er
lengra aftur í tímann.
Punktar sem leggja utan Bollinger bandanna í nokkrum
rammastærðum fá einnig aukið vægi þar sem hver rammi
hækkar gildið.

Ef að í tímalínunni er tímabil sem inniheldur svipaðar
eða eins tölur á kafla sem er stærri en rammastærðin er
vandamál að Bollinger 
böndin þrengjast mikið. Það gerði það að verkum að
litlar breytingar eftir þannig tímabil voru gefin
óeðlilega hátt gildi, 
jafnvel töluvert hærra gildi en það sem teldist til
mikið áhugaverðari punkta í sjón. Til að koma í veg
fyrir þetta er notast við
einfalda athugun sem skoðar hvort bandbreiddin fyrir
viðkomandi punkt er óeðlilega þröng miðað við
meðalbandbreidd tímalínunnar.
Ef svo er er punktinum gefið lægra gildi. 

{
  \color{red}
Þegar þetta er skrifað er notast við fasta margföldun en hugmyndir
eru uppi um að lækka 
gildið útfrá meðaltali gilda í tímalínunni.
}

Að lokum hreinsum við flögg sem liggja hlið við hlið,
liggja sömu megin Bollinger bandanna og innihalda
sífellt hækkandi gildi, 
þannig skilum við einungis toppum á hækkunum sem kerfið
sér, en ekki punkta innan hækkunaninnar.

\section{Horft til framtíðar}
Út frá niðurstöðu okkar prófana eru þeir punktar sem kerfið metur
áhugaverða vissulega áhugaverðir en það er þar með ekki sagt að
sú aðferð sem við völdum virki á allar tegundir tímalína. 

Sem dæmi má nefna línur sem innihalda mikinn fjölda punkta.
Líta má á þær tímalínur af 'hærri' upplausn en þær sem innihalda fáa punkta. Í
þeim getur kerfið okkar verið að flagga punkta sem vissulega eru mjög
áhugaverðir séu þeir skoðaðir í því samhengi sem kerfið okkar sér þá þ.e. innan
rammans. Sé ramminn lítill er hægt að bera það saman við að kerfið sé að skoða
tímalínuna úr lítilli fjarlægð. Aftur á móti sé sama tímalína skoðuð úr meiri
fjarlægð getur línan virst svo gott sem flöt og því í heildina ansi óáhugaverð.

Í öðru dæmi væri hægt að hugsa sér tímalínu sem hækkar töluvert upp í ákveðinn
hápunkt og taki svo að lækka aftur niður í fyrri gildi. Gerist þetta með nógu
litlum breytingum á milli punkta er ólíklegt að kerfið okkar sjái nokkuð
athugavert við þetta þó að mögulega hafi eitthvað merkilegt átt sér stað. 

Í þessum tilvikum þyrfti að skoða tímalínurnar í öðru samhengi: út frá því hvort
það eigi sér stað breyting í hinu almenna ferli tímalínunnar. Bollinger böndin
eru ekki mjög gagnleg til þess þar sem að útkoman byggir að miklu leiti á hvaða
rammastærð er notuð en við höfum ekki fundið neina aðferð sem gagnast við að
finna hentuga rammastærð út frá gögnum í tímalínunni.

Fjallað verður um aðferðir sem gagnast við að skoða þessa eiginleika í kaflanum
'Horft til Framtíðar'

\subsection{Linear Regression}
Í þessari aðferð má hugsa sér að reynt sé að draga eins fáar beinar línur í
gegnum almenna ferli tímalínunnar. Sé hægt að gera það með einni línu má líta
svo á að engin stórvægileg breyting eigi sér stað í almenna ferlinu og því í
heild sinni sé tímalínan óáhugaverð. Sé ekki hægt að gera það með einni línu
hefur átt sér stað nægileg breyting og má þá líta á skurðpunkt línanna sem
hugsanlega áhugaverðan punkt í tímalínunni.

\subsection{Fourier Transformation}
Eins og fjallað var um í kaflanum Rannsóknir nýtist Fourier
Transformation við að finna endurtekningar í tímalínum með því að brjóta þær
niður í tíðnir sínar.  
Einnig væri ekki einungis hægt að finna hvort það eigi sér stað endurtekning í
tímalínunni, heldur hvar. Þannig væri mögulegt að skoða endurtekningarnar nánar,
t.d. út frá því hvort ein endurtekningin sé óeðlilega há eða lág, eða hvort hana
hreinlega vanti eitt árið.


\subsection{Niðurstaða}
Á vissum tímapúnktu í verkefninu varð okkur ljóst að Bollinger bönd hentuðu
mjög vel fyrir mest af því sem við óskuðum að kerfið skyldi gera.
Því var sett aukin áhersla á að prófa mismunandi útfærslur Bollinger banda og dregið
úr prófunum annarra aðferða.

\hfil \\
\hfil \\
\hfil \\
\hfil \\









Upprunalega var kerfið okkar hugsa þannig að það myndi athuga hvort ný gögn sem
væri verið að bæta við gangasett Datamarket væru áhugaverð eður ei. Hinsvegar
varð okkur ljóst í þróunnarferlinu að kerfið okkar gæti gert meira en unnið
eingöngu með nýjustu uppfærslurnar og varð því kerfið okkar yfirgripsmeira en
upprunalega var hugsað.

Datamarket mun geta nýtt kerfið okkar bæði til að finna áhugaverðust atburðina
út úr tilteknum hópi gagnasetta sem og keyrt það á nýjustu uppfærslurnar.

TAKA ÚT Markmiðið með þessu verkefni er að greina þessi tilvik sjálfkrafa þannig
að hægt sé að beina athygli notenda DataMarket að þeim. Þannig mætti t.d. hugsa
sér að dregnir yrðu fram á forsíðu DataMarket tenglar á þróun sem kerfið telur
um markverða atburði að ræða.


{
  \color{red}
TAKA ÚT Markmiðið með þessu verkefni er að greina þessi tilvik sjálfkrafa þannig að hægt sé að beina athygli notenda DataMarket að þeim. Þannig mætti t.d. hugsa sér að dregnir yrðu fram á forsíðu DataMarket tenglar á þróun sem kerfið telur um markverða atburði að ræða.
}








\end{document}