\documentclass{article}
\usepackage[icelandic]{babel}
\usepackage[T1]{fontenc}
\usepackage[utf8]{inputenc}
\usepackage{graphicx}
\usepackage{wrapfig}
\usepackage{subfig}
\usepackage{float}

%Leitað er af myndum í eftirfarandi möppum
\graphicspath{{./technical/}}

\begin{document}

\tableofcontents
\newpage

\section{Inngangur}
Verkefnið snérist um að búa til kerfi sem greinir
breytingar á gagnasettum í kerfum DataMarket og flagga
frávik frá eðlilegri þróun tiltekinnar tímalínu, sem
gætu bent til þess að um áhugaverðan atburð sé að ræða.

Fyrirhugað er að kerfið okkar verði keyrt innan kerfis
Datamarket og er hugsa þannig að hægt sé að keyra það á
gögnin þeirra hvenær sem er.

Upprunalega var kerfið okkar hugsa þannig að það myndi
athuga hvort ný gögn sem væri verið að bæta við
gangasett Datamarket væru áhugaverð eður ei. Hinsvegar
varð okkur ljóst í þróunnarferlinu að kerfið okkar gæti
gert meira en unnið eingöngu með nýjustu uppfærslurnar
og varð því kerfið okkar yfirgripsmeira ein upprunalega
var hugsað.

\section{Lýsing verkefnis}
Skrifa almennt um Datamarket

Hvert gagnasett hjá DataMarket inniheldur tímaraðir sem
sýna þróun tölulegra stærða yfir tíma. Á hverjum degi
eru tugir, eða jafnvel hundruð slíkra gagnasetta
uppfærð í kerfinu. Hvert gagnasett inniheldur svo að
lágmarki eina, en allt að nokkurþúsund tímaraðir. Sú
þróun sem ein tímaröð sýnir getur verið allt frá vel
þekktum stærðum, s.s. verðbólgu, atvinnuleysi eða
hitastigi í Reykjavík, til mjög sérhæfðra eða jafnvel
undarlegra hluta, eins og innflutningsverðmæti
leðurvara frá Brasilíu!

Mikið er fylgst með þessum algengustu stærðum og
stórvægilegar breytingar í þeim rata iðulega í fréttir
umsvifalaust. En mjög markverð, áhugaverð eða jafnvel
varasöm þróun getur líka birst í stærðum sem fáir
fylgjast með og enginn tekur etv. eftir.

Markmiðið með þessu verkefni er að greina þessi tilvik
sjálfkrafa þannig að hægt sé að beina athygli notenda
DataMarket að þeim. Þannig mætti t.d. hugsa sér að
dregnir yrðu fram á forsíðu DataMarket tenglar á þróun
sem kerfið telur um markverða atburði að ræða.

Verkefnið snýst s.s. um að útbúa aðferðafræði sem er
líkleg til að grípa tilvik af þesu tagi. Tiltölulega
auðvelt er að útbúa mjög gróf algrím sem myndu grípa
mörg tilvik, en líklega einnig leiða af sér töluvert
magn “false positives”. Hugsunin er að byrja á að útbúa
slíka einfalda frumgerð og endurbæta svo aðferðafræðina
eftir því sem reynsla, hugmyndaauðgi og tími gefa
tilefni til.

\newpage

\section{Sprettir}


\section{TækniSkýrsla}


\subsection{Fourier transformations} 

Fourier umbreytingar eru stærðfræðilegar aðgerðir
sem brjóta merki niður í tíðnir þess, slíkt gátum við nýtt okkur með því að láta
umbreyta tímalínu í tíðnir. 
Ef ákveðin tíðni er mjög áberandi í tíðniritinu, þá er að öllum líkindum endurteknging með þeirri tíðni í tímalínunni. 
Sem dæmi um slíkt er tímalínan á mynd~\ref{fig:rafmagnsnotkun} sem sýnir rafmagnsnotkun í ótilgreindu landi.


\begin{figure}[H]
  \centering
    \subfloat[Tímalína]{\label{fig:rafmagnsnotkun}\includegraphics[width=0.42\textwidth]{Rafmagnsnotkun}}                    
    \subfloat[Tíðnirit]{\label{fig:fouriergraph}\includegraphics[width=0.42\textwidth]{Fourier}} 
  \caption{Fourier Transformations}
  \label{fig:fourier}
\end{figure}

Það má sjá að tímalínan hefur ákveðnar endurtekningar,
nánar til tekið þá er mikil notkun á veturna en lítil á
sumrin.
Þessa föstu sveiflu í tímalínunni greinir Fourier og
skilar grafi á mynd~\ref{fig:fouriergraph}. 
Úr því er síðan hægt að greina að endurtekning á sér stað með 12 staka millibili.





\subsection{Hlaupandi meðaltal (HM)}
Í tölfræði er hlaupandi meðaltal, einnig kallað
fljótandi meðaltal, notað til greiningar á gagnamengi. 
Er það gert með því að mynda röð meðaltala úr
mismunandi hlutmengjum heildarmengisins.
Ef gefin er röð $Z$ talna og ákveðin stærð ramma
(hlutmengis) er $N$, þá er fljótandi meðaltal fundið
með því að fyrst reikna meðaltal 
talna úr sæti $0,1,\dots,N-1$. Þá er ramminn færður
fram um eitt sæti og meðaltal fundið af tölum í sætum
$1,2,\dots,N$. 
Þessi aðgerðarröð er endurtekin yfir alla talnaröðina,
$Z-N$ sinnum.  \\

$HM = \frac{x+(x+1)+\dots+(x+(N-1))}{N}$
\\
Línan sem tengir svo saman öll meðaltölin er hið
fljótandi meðaltal þar sem hver punktur á línunni
samsvarar 
meðaltali viðkomandi hlutmengis í heildarmengi ganganna
sem verið er að meta. 
 
\subsection{Vegið hlaupandi meðaltal (VHM)}
Hlaupandi meðaltal getur einnig notað ójöfn gildi fyrir
hvert stak á línunni.
Þá er um vegið meðaltal að ræða og er meðaltalinu gefin
einhver margfeldisáhrif eftir því hvar á línunni það er
staðsett. 
Það vægi breytist línulega. Summu margfeldi allra
gildana í tilteknum ramma er svo deilt með summu allra
mergfeldanna. 
Ef um ramma af stærð 10 er að ræða og gildin eru
margfölduð eftir sætisröð, þá fæst: \\
$VHM =
\frac{x+2(x+1)+3(x+2)+\dots+(10(N-1))}{1+2+\dots+10}$
\subsection{Bollinger bands}
\label{sec:third}
Bollinger bönd eru upprunalega þróuð sem greiningartól
á þróun verðbréfaverða. 
Tilgangur þeirra er að veita viðeigandi skilgreiningu
háum og lágum gildum. Einnig hefur þessari aðferð verið
beitt á ýmis önnur
vandamál með misgóðum niðurstöðum. \\
Bollinger bönd samanstanda af
\begin{itemize}
  \item Miðband, $N-lotu$ einfalt hlaupandi meðaltal.
  \item Efra band, $N-lotu$ staðalfrávik margfaldað með
$K$, fyrir ofan miðbandið ($HM + K\sigma$).
  \item Neðra band, $N-lotu$ staðalfrávik margfaldað
með $K$, fyrir neðan miðbandið ($HM -
K\sigma$).%\\\\\\\\\\\\\\\\\
\end{itemize}

\begin{wrapfigure}{l}{6cm}
%\begin{figure}
 \begin{center}
 \includegraphics[width=.45\textwidth]{Bollinger.png}
 \caption{Rauðu strikalínurnar sýna Bollinger böndin}
  \end{center}
%\end{figure}
\end{wrapfigure}
\hfill
\\\\\\\\\\\\\\\\\\\\\\\\\\\\\\\\\\
Bollinger bönd nýtast því vel til greininga hvar í
tímalínu óeðlileg hækkun eða lækkun á sér stað.
\\
\subsubsection{Okkar útfærsla á Bollinger}
Útfærslan okkar byggist á því að ítra í gegnum
mismunandi stórar rammastærðir fyrir hverja tímalínu.
Hver rammastærð gefur hverjum punkti einkunn út frá því
hvar punkturinn er staddur með tilliti til Bollinger
bandanna.
Til þess ákváðum við að nota breytta útgáfu af því sem
kallast $\%b$ sem gefur punkti gildi yfir 1 ef
punkturinn er fyrir utan 
efri mörk Bollinger bandsins, tölugildi á bilinu 1-0 ef
punkturinn liggur á milli þeirra og neikvæða tölu ef
hann er undir þeim.
$\%b={(last-lowerBB)}{(upperBB-lowerBB)}$
Við notum breytta útgáfu af formúlunni sem hundsar
hvort við erum fyrir ofan eða neðan böndin og gefur
einfaldlega gildi á bilinu
0-1 ef punkturinn liggur á milli bandanna en annars
gildi yfir einum ef punkturinn er fyrir utan. Haldið er
utan um hvort punkturinn 
lá fyrir ofan eða neðan í flagginu sjálfu.
$\%b=\frac{abs((item - avg[index])}{(upperlim[index] -
avg[index]))}$
Hver rammastærð margfaldar sína útkomu úr formúlunni
við gildið sem punkturinn hefur fyrir.
Þannig gefum við punktum sem líta út fyrir að vera
snörp hækkun í litlum ramma lækkað vægi þegar hann
lendir innan bandanna í stærri ramma
og telst því til eðlilegrar hegðunar þegar horft er
lengra aftur í tímann.
Punktar sem leggja utan Bollinger bandanna í nokkrum
rammastærðum fá einnig aukið vægi þar sem hver rammi
hækkar gildið.

Ef að í tímalínunni er tímabil sem inniheldur svipaðar
eða eins tölur á kafla sem er stærri en rammastærðin er
vandamál að Bollinger 
böndin þrengjast mikið. Það gerði það að verkum að
litlar breytingar eftir þannig tímabil voru gefin
óeðlilega hátt gildi, 
jafnvel töluvert hærra gildi en það sem teldist til
mikið áhugaverðari punkta í sjón. Til að koma í veg
fyrir þetta er notast við
einfalda athugun sem skoðar hvort bandbreiddin fyrir
viðkomandi punkt er óeðlilega þröng miðað við
meðalbandbreidd tímalínunnar.
Ef svo er er punktinum gefið lægra gildi. Þegar þetta
er skrifað er notast við fasta margföldun en hugmyndir
eru uppi um að lækka 
gildið útfrá meðaltali gilda í tímalínunni.

Að lokum hreinsum við flögg sem liggja hlið við hlið,
liggja sömu megin Bollinger bandanna og innihalda
sífellt hækkandi gildi, 
þannig skilum við einungis toppum á hækkunum sem kerfið
sér, en ekki punkta innan hækkunaninnar.






















\end{document}