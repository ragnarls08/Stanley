\documentclass{article}
\usepackage[icelandic]{babel}
\usepackage[T1]{fontenc}
\usepackage[utf8]{inputenc}
\linespread{1.5}
\begin{document}

\tableofcontents
\newpage

\section{Inngangur}
Verkefnið snérist um að búa til kerfi sem greinir breytingar á gagnasettum í kerfum DataMarket og flagga frávik frá eðlilegri þróun tiltekinnar tímalínu, sem gætu bent til þess að um áhugaverðan atburð sé að ræða.

\section{Lýsing verkefnis}
Sú þjónusta sem Datamarket veitir er að taka sama töluleg gögn frá opinberum stofnunum og einkaaðilum og gerir þau aðgengileg á einum stað sem og birta þau á auðskiljanlegan máta. Gagnsafn Datamarket er gríðarstórt og fer ört vaxandi. Frá og með 25. janúar 2011 samanstóð það af meira en 13.000 gagnasettum sem innihélt næstum 100 milljón tímalínur. Það gefur augaleið að það er ógerlegt að fara handvirkt í gegnum þetta magn af gögnum til að finna áhugaverða atburði því ætti kerfið okkar að vera kærkomin viðbót við kerfi Datamarket.

Hvert gagnasett hjá DataMarket inniheldur tímaraðir sem sýna þróun tölulegra stærða yfir tíma. Á hverjum degi eru tugir, eða jafnvel hundruð slíkra gagnasetta uppfærð í kerfinu. Hvert gagnasett inniheldur svo að lágmarki eina, en allt að nokkurþúsund tímaraðir. Sú þróun sem ein tímaröð sýnir getur verið allt frá vel þekktum stærðum, s.s. verðbólgu, atvinnuleysi eða hitastigi í Reykjavík, til mjög sérhæfðra eða jafnvel undarlegra hluta, eins og innflutningsverðmæti leðurvara frá Brasilíu!

Mikið er fylgst með þessum algengustu stærðum og stórvægilegar breytingar í þeim rata iðulega í fréttir umsvifalaust. En mjög markverð, áhugaverð eða jafnvel varasöm þróun getur líka birst í stærðum sem fáir fylgjast með og enginn tekur etv. eftir.

Upprunalega var kerfið okkar hugsa þannig að það myndi athuga hvort ný gögn sem væri verið að bæta við gangasett Datamarket væru áhugaverð eður ei. Hinsvegar varð okkur ljóst í þróunnarferlinu að kerfið okkar gæti gert meira en unnið eingöngu með nýjustu uppfærslurnar og varð því kerfið okkar yfirgripsmeira en upprunalega var hugsað.

Datamarket mun geta nýtt kerfið okkar bæði til að finna áhugaverðust atburðina út úr tilteknum hópi gagnasetta sem og keyrt það á nýjustu uppfærslurnar.

TAKA ÚT Markmiðið með þessu verkefni er að greina þessi tilvik sjálfkrafa þannig að hægt sé að beina athygli notenda DataMarket að þeim. Þannig mætti t.d. hugsa sér að dregnir yrðu fram á forsíðu DataMarket tenglar á þróun sem kerfið telur um markverða atburði að ræða.

\newpage

\section{Sprettir}






\end{document}