\documentclass{article}
\usepackage[icelandic]{babel}
\usepackage[T1]{fontenc}
\usepackage[utf8]{inputenc}

\begin{document}

\tableofcontents
\newpage

\section{Inngangur}
Verkefnið snérist um að búa til kerfi sem greinir breytingar á gagnasettum í kerfum DataMarket og flagga frávik frá eðlilegri þróun tiltekinnar tímalínu, sem gætu bent til þess að um áhugaverðan atburð sé að ræða.

Fyrirhugað er að kerfið okkar verði keyrt innan kerfis Datamarket og er hugsa þannig að hægt sé að keyra það á gögnin þeirra hvenær sem er.

Upprunalega var kerfið okkar hugsa þannig að það myndi athuga hvort ný gögn sem væri verið að bæta við gangasett Datamarket væru áhugaverð eður ei. Hinsvegar varð okkur ljóst í þróunnarferlinu að kerfið okkar gæti gert meira en unnið eingöngu með nýjustu uppfærslurnar og varð því kerfið okkar yfirgripsmeira ein upprunalega var hugsað.

\section{Lýsing verkefnis}
Skrifa almennt um Datamarket

Hvert gagnasett hjá DataMarket inniheldur tímaraðir sem sýna þróun tölulegra stærða yfir tíma. Á hverjum degi eru tugir, eða jafnvel hundruð slíkra gagnasetta uppfærð í kerfinu. Hvert gagnasett inniheldur svo að lágmarki eina, en allt að nokkurþúsund tímaraðir. Sú þróun sem ein tímaröð sýnir getur verið allt frá vel þekktum stærðum, s.s. verðbólgu, atvinnuleysi eða hitastigi í Reykjavík, til mjög sérhæfðra eða jafnvel undarlegra hluta, eins og innflutningsverðmæti leðurvara frá Brasilíu!

Mikið er fylgst með þessum algengustu stærðum og stórvægilegar breytingar í þeim rata iðulega í fréttir umsvifalaust. En mjög markverð, áhugaverð eða jafnvel varasöm þróun getur líka birst í stærðum sem fáir fylgjast með og enginn tekur etv. eftir.

Markmiðið með þessu verkefni er að greina þessi tilvik sjálfkrafa þannig að hægt sé að beina athygli notenda DataMarket að þeim. Þannig mætti t.d. hugsa sér að dregnir yrðu fram á forsíðu DataMarket tenglar á þróun sem kerfið telur um markverða atburði að ræða.

Verkefnið snýst s.s. um að útbúa aðferðafræði sem er líkleg til að grípa tilvik af þesu tagi. Tiltölulega auðvelt er að útbúa mjög gróf algrím sem myndu grípa mörg tilvik, en líklega einnig leiða af sér töluvert magn “false positives”. Hugsunin er að byrja á að útbúa slíka einfalda frumgerð og endurbæta svo aðferðafræðina eftir því sem reynsla, hugmyndaauðgi og tími gefa tilefni til.

\newpage

\section{Sprettir}






\end{document}