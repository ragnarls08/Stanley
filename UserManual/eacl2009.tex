%
% File eacl2009.tex
%
% Contact  oflazer@gmail.com or das@ling.uni-potsdam.de
%%

%% Based on the style files for EACL 2006 by 
%%e.agirre@ehu.es or Sergi.Balari@uab.es
%% and that of ACL 08 by Joakim Nivre and Noah Smith

\documentclass[11pt]{article}
\usepackage{eacl2009}
\usepackage{times}
\usepackage{url}
\usepackage{latexsym}
\usepackage[icelandic]{babel}
\usepackage[T1]{fontenc}
\usepackage[utf8]{inputenc}

%\setlength\titlebox{6.5cm}    % You can expand the title box if you
% really have to

\title{Rýnir - Notendahandbók}


\date{}

\begin{document}
\maketitle

\section{Inngangur}
Rynir er tölfræðigreinir sem hannaður er fyrir Datamarket.com.

Datamarket er gagnagátt sem veitir aðgengi að tölfræði og skipulögðum gögnum frá ýmsum stofnunum bæði opinberum og úr einkageiranum.

Tilgangur Rýnirs er að greina tiltekið safn af tímröðum og skila af sér flöggum sem segja til um áhugaverða atburði í tíma fyrir gefið safn.
 
\section{Kröfur}
Rýnir er skrifaður í Python 2.6 og styðst við eftirfaradi aðgerðasöfn (e. libraries)\\
\renewcommand{\theenumi}{\roman{enumi}}
\renewcommand{\labelenumi}{\theenumi}
\begin{enumerate}
 \item NumPy, útgáfa 1.5.1 eða nýrra\\ \url{http://numpy.scipy.org/}
 \item SciPy, útgáfa 0.7.0 eða nýrra\\ \url{http://www.scipy.org/}
 \item Mathplotlib, útgáfa 0.99.1.1 eða nýrra\\ \url{http://matplotlib.sourceforge.net/index.html}
 \item SciKits.timeseries, útgáfa 0.91.3 eða nýrra\\ \url{http://pytseries.sourceforge.net/index.html}
 \item Unittest2, útgáfa 0.5.1 eða nýrra\\ \url{http://pypi.python.org/pypi/unittest2}
 \item Simplejson, útgáfa 2.0.9 eða nýrra\\ \url{http://pypi.python.org/pypi/simplejson/}
\end{enumerate}

\section{Almennar leiðbeiningar}


\subsection{Innsetning}
Engra sérstakara aðgerða er þörf ef allar kröfur eru uppfylltar. Afþjappið skráarpakkanum á þeim stað sem hentar hverju sinni.
\subsection{Stillingar}
Í stillingaskrá (e.config) er hægt að stilla eftirfarandi breytur;\\

\begin{itemize}
  \item Url: Slóðin á þann API sem skal tengjast á
  \item Key: Aðgangslykill að API

  \item ConsolidateFlags: Gefur þann möguleika að skila einungis áhugaverðasta flagginu af röð samliggjandi flagga
  \item Filter: Segir til um hversu hátt áhugaverðaleika gildi flagg þarf svo því sé skilað

  \item K: Ákveður efri og neðri mörk sem K sinnum N-lotu staðalfrávik þannig að efri mörk eru $ K*\sigma$ og neðri mörk $-(K*\sigma)$
  \item FrameSize: Ákveður rammastærð eða N-lotu einfalds fljótandi meðaltals (e. simple moving average)
  \item MaxTimelineSize: Ef fjöldi punkta í tímalínu er stærri en MaxTimelineSize þá er sú tímalína ekki send til greiningar
  \item NumberOfThreads: Segir til um hversu marga þræði kerfið hefur leyfi til að nota

  \item TopN: Ákveður hversu mörgum flöggum er skilað. Skilar TopN áhugaverðustu flöggunum
  \item NMostRecent: Segir til um hvort skila eigi niðurstöðu fyrir alla tímalínuna eða einungis skoða NMostRecent nýjustu punktanna
\end{itemize}

Til að fulla greiningu og óskert gögn skal stilla bæði TopN og NMostRecent á 0.

\subsection{Notkun}
Í Rýni er hægt að kalla á fallið analyze() og senda með því lista af fyrirspurnarstrengjum (e.query strings).
Nánari útlistun á fyrirspurnarstrengjum Datamarket er að finna á síðu þeirra \url{http://datamarket.com/api/v1/}.

Kallið í analyze() skilar lista af flöggum. Hvert flagg inniheldur upplýsingar um úr hvaða gagnasafni og 
tímalínu viðkomandi flagg er ásamt staðsetningu innan tímalínu.

\end{document}
