%
% File eacl2009.tex
%
% Contact  oflazer@gmail.com or das@ling.uni-potsdam.de
%%

%% Based on the style files for EACL 2006 by 
%%e.agirre@ehu.es or Sergi.Balari@uab.es
%% and that of ACL 08 by Joakim Nivre and Noah Smith

\documentclass[11pt]{article}
\usepackage{eacl2009}
\usepackage{times}
\usepackage{url}
\usepackage{latexsym}
\usepackage[icelandic]{babel}
\usepackage[T1]{fontenc}
\usepackage[utf8]{inputenc}

%\setlength\titlebox{6.5cm}    % You can expand the title box if you
% really have to

\title{Rýnir}

\author{Arnór Barkarson\\
  Reykjavík University\\
  Reykjavík, Iceland\\
  {\tt arnorbarkar@ru.is}  \And
  Ragnar Lárus Sigurðsson\\
  Reykjavík University\\
  Reykjavík, Iceland\\
  {\tt  ragnarls08@ru.is}\\  \And
  Þórður Arnarsson\\
  Reykjavík University\\
  Reykjavík, Iceland\\
  {\tt  thordura08@ru.is}  \And
  Gunnar Sveinsson\\
  Reykjavík University\\
  Reykjavík, Iceland\\
  {\tt  gunnarsve06@ru.is} 
}

\date{}

\begin{document}
\maketitle

\section{Introduction}
Rynir is a statistical analizer designed for Datamarket.com, 
a data portal that provides access to statistics and structured data from various public and private sector organizations.\\ 
It's purpose is to analyze time series and return a list of flags, 
indicating intresting points in time for a given set of time series. 

 
\section{Dependencies}
Rýnir is written in Python 2.6 and has the following dependencies:\\
\renewcommand{\theenumi}{\roman{enumi}}
\renewcommand{\labelenumi}{\theenumi}
\begin{enumerate}
 \item NumPy\\ \url{http://numpy.scipy.org/}
 \item SciPy\\ \url{http://www.scipy.org/}
 \item Mathplotlib\\ \url{http://matplotlib.sourceforge.net/index.html}
 \item SciKits.timeseries\\ \url{http://pytseries.sourceforge.net/index.html}
 \item Unittest2\\ \url{http://pypi.python.org/pypi/unittest2}
 \item Simplejson\\ \url{http://pypi.python.org/pypi/simplejson/}
\end{enumerate}

\section{General Instructions}


\subsection{Installation}
No special oprerations are needed if all dependencies are fulfilled. Simply unzip the package in any directory.
\subsection{Configuration}
In the config file following parameters can be set;\\
N: Determines the framesize or N-period of a simple moving average.\\
K: Determines upper and lower bands at K times an N-period standard deviation, upper = $ K*\sigma$ and lower = $-(K*\sigma)$\\
TOP: Determines how many flags are to be returned at most.

\subsection{Addons}

To be added ;)

\end{document}
