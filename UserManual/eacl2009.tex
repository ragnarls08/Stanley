%
% File eacl2009.tex
%
% Contact  oflazer@gmail.com or das@ling.uni-potsdam.de
%%

%% Based on the style files for EACL 2006 by 
%%e.agirre@ehu.es or Sergi.Balari@uab.es
%% and that of ACL 08 by Joakim Nivre and Noah Smith

\documentclass[11pt]{article}
\usepackage{eacl2009}
\usepackage{times}
\usepackage{url}
\usepackage{latexsym}
\usepackage[icelandic]{babel}
\usepackage[T1]{fontenc}
\usepackage[utf8]{inputenc}

%\setlength\titlebox{6.5cm}    % You can expand the title box if you
% really have to

\title{Rýnir - Notendahandbók}


\date{}

\begin{document}
\maketitle

\section{Inngangur}
Rynir er tölfræðigreinir sem hannaður er fyrir Datamarket.com.

Datamarket er gagnagátt sem veitir aðgengi að tölfræði og skipulögðum gögnum frá ýmsum stofnunum bæði opinberum og úr einkageiranum.

Tilgangur Rýnirs er að greina tiltekið safn af tímröðum og skila af sér flöggum sem segja til um áhugaverða atburði í tíma fyrir gefið safn.
 
\section{Kröfur}
Rýnir er skrifaður í Python 2.6 og styðst við eftirfaradi aðgerðasöfn (e. libraries)\\
\renewcommand{\theenumi}{\roman{enumi}}
\renewcommand{\labelenumi}{\theenumi}
\begin{enumerate}
 \item NumPy\\ \url{http://numpy.scipy.org/}
 \item SciPy\\ \url{http://www.scipy.org/}
 \item Mathplotlib\\ \url{http://matplotlib.sourceforge.net/index.html}
 \item SciKits.timeseries\\ \url{http://pytseries.sourceforge.net/index.html}
 \item Unittest2\\ \url{http://pypi.python.org/pypi/unittest2}
 \item Simplejson\\ \url{http://pypi.python.org/pypi/simplejson/}
\end{enumerate}

\section{Almennar leiðbeiningar}


\subsection{Innsetning}
Engra sérstakara aðgerða er þörf ef allar kröfur eru uppfylltar. Afþjappið skráarpakkanum á þeim stað sem hentar hverju sinni.
\subsection{Stillingar}
Í stillingaskrá (e.config) er hægt að stilla eftirfarandi breytur;\\
N: Ákveður rammastærð eða N-lotu einfalds fljótandi meðaltals (e. simple moving average).\\
K: Ákveður efri og neðri mörk sem K sinnum N-lotu staðalfrávik þannig að efri mörk eru $ K*\sigma$ og neðri mörk $-(K*\sigma)$.\\
TOP: Ákveður hversu mörgum flöggum er skilað. Skilar TOP áhugaverðustu flöggunum.\\
FULL: Segir til um hvort skila eigi niðurstöðu fyrir alla tímalínuna eða einungis þeirri nýjustu.
\subsection{Notkun}
Í Rýni er hægt að kalla á fallið analyze() og senda með því lista af fyrirspurnarstrengjum (e.query strings).
Nánari útlistun á fyrirspurnarstrengjum Datamarket er að finna á síðu þeirra \url{http://datamarket.com/api/v1/}.

Kallið í analyze() skilar lista af flöggum. Hvert flagg inniheldur upplýsingar um úr hvaða gagnasafni og 
tímalínu viðkomandi flagg er á samt staðsetningu innan tímalínu.

\end{document}
